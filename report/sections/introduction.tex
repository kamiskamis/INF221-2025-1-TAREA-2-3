Para aprovechar mejor los recursos, es importante desarrollar soluciones óptimas analizando y diseñando bien los algoritmos. Este informe compara dos métodos clásicos: fuerza bruta y programación dinámica, aplicados al problema de encontrar diferencias mínimas entre dos secuencias de caracteres. La idea es mostrar la menor cantidad de pares de substrings que representen todas las diferencias. Con esto, buscamos entender cuál es más eficiente según el tamaño de la entrada y por qué es importante elegir el enfoque correcto dependiendo del problema.

\section{Fuerza Bruta}

La fuerza bruta resuelve el problema de la manera más directa posible. Básicamente, revisa todas las formas en las que se pueden alinear los caracteres comunes en las secuencias y de ahí saca las diferencias.

Este método siempre encuentra una solución válida, pero cuando las secuencias son más grandes, se vuelve muy lento. Esto pasa porque repite subproblemas innecesariamente y no guarda resultados intermedios, lo que hace que use demasiado tiempo y memoria.

\section{Programación Dinámica}

La programación dinámica es más eficiente porque almacena resultados de subproblemas ya resueltos. Esto evita cálculos repetidos y aprovecha patrones del problema para ahorrar tiempo.

En general, funciona así:
\begin{enumerate}
    \item Se analiza la estructura del problema para ver qué subproblemas se repiten.
    \item Se usan técnicas recursivas, pero guardando los resultados para no recalcularlos.
    \item A partir de las soluciones parciales guardadas, se arma la respuesta óptima, minimizando las diferencias entre las cadenas.
\end{enumerate}

Este método es más rápido, escalable y útil cuando la entrada tiene mayores dimensiones, por lo que en muchas situaciones es mejor que la fuerza bruta.

