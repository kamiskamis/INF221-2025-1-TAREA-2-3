El método de fuerza bruta busca comparar de forma exhaustiva todas las maneras posibles de alinear los caracteres de dos secuencias $s$ y $t$. Se intenta encontrar el mayor prefijo común y luego dividir las cadenas en partes que representen las diferencias.

Este enfoque explora todas las opciones posibles, pero no evita repetir cálculos innecesarios, lo que hace que la cantidad de operaciones crezca de manera descontrolada a medida que aumenta el tamaño de las secuencias.

\subsubsection{Complejidad}

\begin{itemize}
    \item Tiempo: $O(2^{\min(n,m)})$, debido a la gran cantidad de posibles particiones.
    \item Espacio: $O(d)$, donde $d$ es la profundidad de la recursión.
\end{itemize}


